\documentclass{acm_proc_article-sp}

\usepackage{graphicx}
\usepackage{caption}
\usepackage{subcaption}
\usepackage{enumitem}
\DeclareCaptionType{copyrightbox}

\usepackage{tikz}
\usetikzlibrary{backgrounds, fit, shapes}

\usepackage{hyperref}
\hypersetup
{
	colorlinks=true,
}

\makeatletter
\let\@copyrightspace\relax
\makeatother

\begin{document}

\title{The Code Following: What builds a following on Open Source Software?}
\numberofauthors{2}
\author{
\alignauthor
Jos{\'e} Calvo-Villagr{\'a}n\email{jcalvovi@uwaterloo.ca}
\and
\alignauthor
Madhur Kukreti\email{mkukreti@uwaterloo.ca}
}

\maketitle

\begin{abstract}
Social coding websites have experienced a significant rise in popularity in the past few years and so has the development and use of Open Source Software. GitHub is one of the popular social coding website with more than four million users and six million repositories. For a typical new user joining GitHub it can become a daunting task to interact and grow in this vast online community. This calls for research on what typical behaviours are observed and rewarded in such a diverse network and what useful information can be extracted by analyzing previous trends.  In this paper we try to find the rationale behind why people follow projects on GitHub. We discuss some previous work that throws light on similar subjects. To answer the question, we collect the information of 618,353 users and 868,177 repositories on GitHub. 
\end{abstract}

% A category with the (minimum) three required fields
\category{K.4.3}{Organizational Impacts}{Computer-supported collaborative work}
%A category including the fourth, optional field follows...
\category{D.2.9}{Management}{Programming Teams}

\terms{Human Factors, Contributors, Open Source}

\keywords{Repositories, Social Networking, Software Engineering} % NOT required for Proceedings


\section{Introduction}

With the advent of Web 2.0 online collaboration has become not only a part of internet experience but rather a necessity. People today are not just passive viewers of the online content but rather active contributors. Social networking sites such as Facebook and Linkedin allow users to create both, their personal and professional identities respectively. Further, online communities such as blogs and Wikis allow people to express opinions, carry out discussions, ask questions and answer queries, among other things. In addition, media sharing and hosting websites such as YouTube ensure fast propagation and easy accessibility of digital content. Clearly, digital collaborations have revolutionized the way information is generated (i.e. through multiple sources), stored (social graphs and complex data types) and accessed (fast and ubiquitous) using just a web browser. In this paper we will particularly focus on collaborations between software developers, “Social Coding” and Open Source Software development. We will discuss how these collaborations function by analysing one of the most popular social coding website -GitHub. We would then conduct a study to assess various factors which govern the dynamics of such kind of collaborations.
Recently, platforms such as SourceForge, GitHub, BitBucket, CodePlex have experienced a rise in popularity. These are also referred to as social coding platforms i.e. development environment that encourages formal and informal collaboration on software projects providing opportunities for discussion and sharing. Open source software in particular has experienced a rise in popularity. These open source projects are licensed to be freely modified and distributed and hence are often developed publically where multiple developers collaborate to add or modify functionality and remove bugs. This new style of development is driven by a rather different set of factors than traditional development. A study conducted by John et al [1] shows that Open source development follows a very modular approach where emphasis is on implementation. Unlike in traditional software development where a requirement specification document specifies design and approach, in Open source, design is a secondary concern which is consistently shaped throughout the development lifecycle via discussions (blogs) and mailing lists and is rather a byproduct of the source code. In addition to this there is also a difference how developers interact on social coding platforms. One of the most important characteristic of OSS development is voluntary contribution [2] without monetary incentives . While proprietary software is a subject to investment and returns, open source communities encourage transparency [3] , collaboration and sharing of ideas. From an economic standpoint, generation of original work takes time and effort in OSS but is compensated in low distribution cost which are almost negligible [4] . Most importantly, open source embraces creative freedom, collaboration, voluntary participation and encourages derivative works at minimal or zero access costs. Moreover, these collaborative platforms do not follow a traditional organizational hierarchy. Rather, the social graph structure of these projects introduces several different factors which guide and shape these collaborations. For example, a studies [4] [5] show that reputation of OSS developers plays an important role in attracting contributors to their projects. This is because, in such a setting where there is absence of personal knowledge about other developers for example their educational and professional qualifications, the only way to network with the right people is to indirectly assess them through their online reputation. This social aspect of OSS development provides insight into dynamics of collaborations and open interesting opportunities for research. Several studies throw light on what drives people towards OSS. Some of these factors include personal need for software [], creative stimulation [], learning [], gift-giving intentions [] and significant returns on contribution []. 

\section{Related Work}

A common solution to the imbalance problem is to replicate nodes across partitions.
Solutions for the replication problem in graph databases have been proposed in
SPAR~\cite{Pujol12}, Mondal~\cite{Mondal12} and TAO~\cite{Bronson13}. Pregel
\cite{Malewicz10}, on the other hand, relies on hash partitioning to produce an even
distribution of nodes and assumes it will provide a good probability of balanced access
between partitions. Pregel-like systems, such as Sedge~\cite{Yang12} and
Mizan~\cite{Khayyat13}, analyze workloads between supersteps and can
move/replicate nodes based on access patterns. The problem that we noted with
Pregel based systems is that they work strictly under OLAP workloads.

% TAO
TAO~\cite{Bronson13} is Facebook's geo-distributed data store. It introduces a
simple graph database API that helps developers deal with the social graph in a
more intuitive way than key-value data stores. TAO implements an eventual
consistent replication model and is heavily optimized for reads, achieving over a
billion reads per second. It achieves these goals by trying to answer all requests
straight from main memory. Facebook still uses MySQL for persistent storage, but
relies on two tiers of caching clusters to answer client requests. The two tiers are
divided by followers and leaders. Basically, clients contact followers in order to
satisfy a request. If a cache miss occurs on the follower, it contacts the leader.
The leader can contact directly the underlying MySQL cluster. It is also in charge
of maintaining consistency among followers of a geographic region, by invalidating
out-dated cached copies that followers might have. Both the leaders and the
followers implement LRU as a caching policy.

% SPAR
Pujol et al. introduce SPAR~\cite{Pujol12} as a middle-ware solution that
provides partitioning and replication to enforce local semantics. The graph
model that they exploit is one in which a node has many attributes, and those
attributes get updated. This model is in contrast to the work presented in
LinkBench \cite{Armstrong13}, where nodes are simple (i.e. have only one data
field) and updates are represented by creating new nodes with corresponding
edges. As such, SPAR is a poor candidate for storing online social network data
that follows the storage model used by Facebook, since local semantics can not
scale to that volume of neighbours. Local semantics is also excessive, since a
big volume of neighbouring data becomes cold in time.

% Mondal (SPAR with tau)
Mondal et al.~\cite{Mondal12} improve on the model introduced by SPAR, by only
requiring approximately $\tau \in [0, 1]$ of the total number of neighbours to
be local. They also exploit the read/write patterns of nodes to decide when to
replicate nodes in a different partition. In Facebook's graph, vertices are
generally read-only, so vertices see few updates. However, Mondal's model
assumes that nodes have a variable write frequency. They use this property to
cluster nodes together in replication groups. This clustering of nodes removes
the underlying graph structure, especially when all nodes have an equal write
frequency (in the case of Facebook's workload, the read-only nodes). This
results in clusters of nodes that are replicated together, but which are not
necessarily accessed together, thereby still requiring remote traversals in
order to answer queries.

% SEDGE
Yang et al. developed Sedge \cite{Yang12}, a system that focuses on graph
queries that are long lived and analytical in nature, as opposed to the online
and short lived queries of social networks. They created complimentary
partitions so that a query can start in a safe region where it is unlikely to
jump machine boundaries. Any two complimentary partitions have minimal overlap
in the edges crossing machine boundaries. Additionally, they group nodes
together using a heuristic. They record which of those groups of nodes queries
touch to create dynamic replicas of the data resulting in a safe region that
those queries can start from. They achieved 2,000 requests per second on 3-hop
neighbourhood search on a 30 million vertex, 1 billion edge graph on 31
machines. The primary difference between the system in this paper and Sedge is
that we return all data within the k-hop neighbourhood while a neighbourhood
search will return only one. However, this system was still a natural candidate
for comparison by modifying the neighbourhood search to return all vertices. We
reached out to the authors for a copy of their system but they declined.


\section{Data Collection}
\label{sec:collection}
\begin{table}
\centering
\begin{tabular}{ | c | c | c | c | }
	\hline
	Users & Repositories & Followers & Followings \\ \hline
	Top 10 & 93.4 & 10,231.4 & 81.3 \\ \hline
	Top 100 & 129.1 & 3,203.2 & 70.6 \\ \hline
	Top 1K & 70.8 & 744.5 & 121.0 \\ \hline
	Top 10K & 45.8 & 147.7 & 49.2 \\ \hline
	All & 5.1 & 4.1 & 2.8 \\ \hline
\end{tabular}
\caption{Average repositories, followers and followings for GitHub users. Users are ordered by the number of followers. By Top 10 we mean the 10 most followed users. Same applies for Top 100, Top 1K and so on.}
\label{tbl:topusers}
\end{table}
\begin{table}
\centering
\begin{tabular}{ | c | c | c | c | }
	\hline
	Repos & Stars & Forks & Issues \\ \hline
	Top 10 & 21,564.0 & 5,972.8 & 306.1 \\ \hline
	Top 100 & 8,832.3 & 2,309.7 & 204.3 \\ \hline
	Top 1K & 2,600.1 & 622.2 & 141.0 \\ \hline
	Top 10K & 491.0 & 121.2 & 28.7 \\ \hline
	All & 8.5 & 2.2 & 0.6 \\ \hline
\end{tabular}
\caption{Average stars, forks and issues for GitHub repositories. By Top 10 we mean the 10 repositories that have the biggest number of stars. Same applies to Top 100, Top 1K and so on.}
\label{tbl:toprepos}
\end{table}
\begin{table}
\centering
\begin{tabular}{ | l | c | c | }
	\hline
	\multicolumn{1}{|c|}{Company} & Users in Top 10 & Users in Top 1K \\ \hline
	GitHub & 4 & 41 \\ \hline
	Google & 2 & 23 \\ \hline
	Khan Academy & 1 & 1 \\ \hline
	Linux Foundation & 1 & 1 \\ \hline
	Segment.io & 1 & 3 \\ \hline
	Basecamp & - & 4 \\ \hline
	Heroku & - & 7 \\ \hline
	Twitter & - & 11 \\ \hline
	Others & - & 556 \\ \hline
	Not Specified & 1 & 353 \\ \hline
\end{tabular}
\caption{Number of users that a company has between the Top 10 and Top 1K. It shows, for example, that 2 of the users working for Google are between the 10 most popular. This number grows to 23 when the 1K most popular users are considered.}
\label{tbl:topcompanies}
\end{table}

We used GitHub API v3 \cite{GitHubAPI} to collect the data. The advantages of such API is that it provides a more structured and straightforward access to the data. The disadvantage is that it only allows 5,000 queries per hour per user. An alternative to the API would be to create our own web crawler/scraper, with the risk of our IP getting banned by GitHub. We eventually found a way to paralelize the data collection phase, with the use of several user accounts. We built a tool that, given an initial set of users, would recursively fetch a random subset of his/her followers. For the initial set, we chose the one thousand most popular users, plus some chosen randomly. The same could not be done for the repositories. There is no unified criteria for selecting the most popular repositories: number of forks, number of contributors, number of watchers and so on. We ended up including a random set of repositories, ordered by different criteria.

We collected information of 618,353 users in GitHub. Regarding Q1, Table \ref{tbl:topusers} provides an overview of the typical user in GitHub. For comparison purposes, we decided to group the data by the ranking of the user. This shows, for example, that the average number of followers between the 10 most popular users is almost three times that of the Top 100 users. An interesting trend shown here is that the top users are also following a considerable amount of users, with the top 1K users following an average of 121 users, compared to the 2.8 followings that the average user has. We also collected information of 868,177 repositories. The averages are shown in Table \ref{tbl:toprepos}.

Finally, Table \ref{tbl:topcompanies} shows the companies that employ the most popular users in GitHub. It is interesting to note that the ten most popular users are mostly employed by well-known companies. It also shows that the number of independent users is considerable, with 35.3\% of the first 1,000 most popular users declaring themselves as working on their own or representing themselves.



\section{Results}

\begin{table}
\centering
\begin{tabular}{ | l | c | }
	\hline
	Vertices & 1,000,000 \\ \hline
	Edges & 8,700,950 \\ \hline
	Partitions & 8 \\ \hline
	\# client requests & 100,000 \\ \hline
	\# clients & 8 \\ \hline
	\# threads/client & 10 \\ \hline
\end{tabular}
\caption{Configuration used for our evaluations}
\label{tbl:graphconfig}
\end{table}

The configuration for our evaluations can be seen on
Table~\ref{tbl:graphconfig}. For our evaluations, we make the following
assumptions:
\begin{itemize}[topsep=0pt]
	\item Read-Only workload
	\item All vertices are of the same size
	\item Edges do not occupy space in cache
	\item Cache size is measured by the number of vertices
\end{itemize}

Note that all assumptions can be relaxed. They are issues that we intend to
address on future versions of the platform. We compared three caching
algorithms: LRU, ARC and LCC. We ran tests on each algorithm while varying the
cache size on each partition to the following settings (size given by number of
vertices): 1,000 (0.1\% of the total graph size), 2,500 (0.25\%), 5,000 (0.5\%),
10,000 (1\%) and 25,000 (2.5\%). We then measured the following metrics that
allowed us to compare amongst algorithms: Hit-ratios, average number of
partitions traversed while answering a query and latencies. The following
sections discuss the 2NR queries, that we extensively tested.

We also ran extensive tests for the 2RW query. Caching provided us with up to
$\sim$ 33\% improved response time. However, due to the small set of data
touched by 2RW per request, all of the caches that we tested with, and all of
the sizes per cache, saw uniform results. As such, we omit them from this paper.

\subsection{Cache Hit-Ratios}
\begin{figure*}[t]
	\includegraphics[keepaspectratio, width=\textwidth]
	{./img/btnCacheHits.png}
\caption{Comparison of hit-ratios, between the three algorithms,
as cache size increases on the 2NR Query. The graphs are ARC, LRU, LCC, from
left to right.}
\label{fig:hitratios}
\end{figure*}

We start by comparing hit-ratios amongst the different caching algorithms.
Figure \ref{fig:hitratios} shows the results for the different algorithms. As
expected, all caches improve on hit-ratios as the size of the cache grows. It
is not a real surprise to see LCC under perform compared to others (refer to
section \ref{sec:lccfailure} for more details). ARC has the best hit-ratio, but
by a very little margin.

\subsection{Number of partitions}
\begin{figure}
	\includegraphics[keepaspectratio, width=0.4\textheight]
	{./img/btnRemotePartitionsVisited.png}
\caption{Comparison of the average number of remote partitions that
needed to be accessed in order to satisfy a request from a client.}
\label{fig:partitions}
\end{figure}

Next, we compare the average number of partitions that an individual
query needed to access in order to fully satisfy a request. Figure
\ref{fig:partitions} shows this results. We can see how the average of
partitions directly correlates with hit-ratios. This is because the more
vertices you can find in cache, the less networks calls need to be made.
These results show once again that ARC is performing the best. They also
show some of the gains of caching, especially when compared to no
cache.

\subsection{Latencies}
\begin{figure*}
	\begin{center}
	\includegraphics[keepaspectratio, width=0.8\textwidth]
	{./img/btnClientResponseTime.png}
	\end{center}
\caption{Comparison of latencies, as measured by the client, of the
different caching algorithms for the 3NR Query. The latency for LRU at 1000
vertices was an anomaly due to server usage.}
\label{fig:latencies}
\end{figure*}

Finally, we compare latencies as measured by the client. Figure \ref{fig:latencies}
shows the results for latencies. Caching algorithms clearly have an advantage
over no cache, improving performance by up to $\sim$ 25\%. Considerable gains in
latency were not seen until the cache size reached 10000 vertices, since on
average one query would have a result set of 500 vertices.

\subsection{Reasons for LCC's failure}
\label{sec:lccfailure}
When using a single source query coordinator (refer to section
\ref{sec:singlesourcequerycoord}), we see poorer performance when compared to
the other caching strategies. This is because vertices that are two machine hops
away from the start vertex are less likely to be incident to an edge on the
start vertex's local machine.
\footnote{
	Let $N(v)$ be the neighbours of $v$. Assume $v$ is on partition $P_1$. Let
	$S = |N(v) \cap P_3|$. Then, $P(S \geq x | u \in N(v) \cap P_2) \leq P(S \geq
	x)$ for $x \in \mathbb{N}$, when $P_2 \cap P_3 = \emptyset$.
}
As a result, these vertices are more likely to be
evicted from the cache because they have a lower score. This means that the
vertices that are two hops away are very short lived. Consider also the fact
that 1-hop vertices will usually be a small portion of the result set.

However, we believe LCC will yield good results with query models that
make use of intermediate caches (refer to state sharing coordinators,
sections \ref{sec:statepassingquerycoord} and~\ref{sec:distributedstatequerycoord}).

Some optimizations made to LCC involve the creation of LRU levels when they
are about to be used for the first time. A level is destroyed when it becomes
empty. On the negative side, we expected LCC to produce bigger latencies than
other caching policies since it requires an extra call to database in order to
provide a score to a remote vertex. We will be looking on ways to improve this.

\section{Data and Queries}

We make use of LinkBench for generating our graph.  The probability of a node
having some number of edges follows a Zipfian distribution. This means that the
probability of a node having some degree decreases exponentially as the degree
increases.

% read only
% reads of nodes are Zipfian
We re-use their application to generate our workloads. The workloads
that LinkBench see are after cache. As a result, we attempt to approach a
pre-cache workload by adjusting the parameters to LinkBench's distributions.
Second, we assume a read only workload, because TAO and LinkBench observed
primarily reads. This allows us to focus on cache optimizations. The focus of
this paper is not cache invalidation and isolation. The probability of a user
being read some number of times also follows a Zipfian distribution. What this
means is that the probability of a node being read some number of times
decrease exponentially with respect to the number of reads.

% 2-hop query
% random walk queries
% return all nodes
Since photos, status updates, and people are nodes in the graph, queries requesting
all of the photos that a person's friends posted become two hop. One hop to
determine the person's friends, and a second hop to determine their friends' photos.
As a result, we modify the LinkBench workload generator to perform 2-hop queries.
LinkBench only generates 0-hop queries like: getting a node, getting the edges of a
node, and getting an edge. Additionally, we include random walk queries that
represent a user following edges starting from their starting node.

% Result limit
% no history
LinkBench limits the number of nodes returned based on their timestamp. The
reason is that a person would be interested in the most recent events. Since
less than 3\% of queries returned more than 100 vertices, we limit the number of
vertices returned to 100. Secondly, after retrieving the initial vertices, the
probability of reading historically for old vertices was 0.96\%. We removed
these from our workload because a small amount of queries asked for it.


\section{Conclusions}
\label{sec:conclusions}

Lorem ipsum dolor sit amet, consectetur adipiscing elit. Sed eget purus in mi fringilla faucibus ut sit amet enim. Integer ac aliquam lacus. Sed semper luctus leo, a porttitor ligula. Praesent elementum placerat mi. Maecenas suscipit convallis justo, eget aliquam neque blandit vitae. Sed tincidunt dui nec metus sagittis, a semper nibh luctus. Pellentesque felis tortor, interdum sed congue sed, bibendum vitae eros. Suspendisse fringilla bibendum pretium. Vivamus at eros ac urna consectetur auctor.

Vestibulum eros diam, ornare et pulvinar nec, facilisis eget augue. Donec non nulla ac tortor semper bibendum. Donec non dolor dui. Aliquam erat volutpat. Nam orci mauris, ultrices non lobortis ut, pretium luctus sem. Aenean eros lectus, lacinia et sodales lobortis, laoreet vitae nulla. Vestibulum cursus magna dolor, non aliquam massa semper a. Phasellus elementum blandit orci, et viverra mi porttitor sed. Donec pharetra risus quis quam mattis, ac lacinia augue gravida. Vestibulum ante ipsum primis in faucibus orci luctus et ultrices posuere cubilia Curae; Pellentesque sodales magna et orci accumsan interdum. Cras ut metus lorem. Sed aliquet leo sed nisi placerat, adipiscing fringilla ipsum pellentesque.

\section{References}
\label{sec:references}

\bibliographystyle{plain}
\bibliography{paper}
\end{document}
