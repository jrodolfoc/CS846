
\section{Related Work}
\label{sec:related}

[15] investigates the network structure of social coding in GitHub. In this study, they try to measure the strength of a) relationships among different projects and b) relationship between the developers. For this, they create network graphs of 30,000 projects randomly sampled from a set of 1,00,000 projects returned by the GitHub API and then randomly select 30,000 developers from these projects. This data is used to create two networks - A project-project network and a developer-developer network. In the project-project network each node of the network is a project and two projects are connected if they have at least one common developer. A weight is associated to each edge which signifies the number of developers that work on both the projects. In similar way the developer-developer network is constructed where nodes represent the developers and an edge is drawn between two developers when they work together on at least one project. Weight in this network represents the number of projects the two developers have worked upon together. These graphs are then analysed on three network characteristics - 1) Node degree 2) Network diameter and 3) Average shortest path using methodologies mentioned in  [14]. The authors also use the PageRank algorithm on these networks to find out the most influential projects and developers. The results show that the project networks are more interconnected than human networks and that social coding enables substantially more collaboration among developers.

In [3], the effect and value of ``transparency'' on people's inferences and decisions is analysed. This study is based on [13], in which it is claimed that ``awareness about the activities of user behavior on a social website enables other users on the network to draw inferences''. This study was a series of 24 semi-structured interviews with GitHub users where they were asked about the inferences they make about a user or a project and whether these inferences influence the projects or the people they follow. As expected, a rich set of inferences were made mostly based on 4 visual cues namely volume of activity, sequence of actions over time, attention to artifacts and people and the detail of information about an action. Each of these cues revealed a different type of information to the user making the inferences. For example, one of the users said that commits on a project revealed direction and intent of the contributor.