
\section{Related Work}
\label{sec:related}

The authors at \cite{LlanosC12} show that the open style of development is driven by a rather different set of factors than traditional development. They conclude that open source development follows a very modular approach, where emphasis is on implementation. Unlike traditional software development, where a requirement specification document specifies design and approach, in open source design is a secondary concern which is consistently shaped throughout the development lifecycle via discussions (blogs) and mailing lists, and is rather a byproduct of the source code. In addition to this, there is also a difference in how developers interact on social coding platforms.

One of the most important characteristic of OSS development, according to \cite{Hippel2003}, is voluntary contribution without monetary incentives. While proprietary software is a subject to investment and returns, open source communities encourage transparency \cite{Dabbish2012}, collaboration and sharing of ideas. From an economic standpoint, generation of original work takes time and effort in OSS but is compensated by low distribution cost which are almost negligible \cite{Lerner2002}. Most importantly, open source embraces creative freedom, collaboration, voluntary participation and encourages derivative works at minimal or zero access costs. Moreover, these collaborative platforms do not follow a traditional organizational hierarchy. Rather, the social graph structure of these projects introduces several different factors which guide and shape these collaborations. For example, \cite{Lerner2002} \cite{Raymond1998} show that reputation of OSS developers plays an important role in attracting contributors to their projects. This is because, in such a setting there is absence of personal knowledge about other developers, for example their educational and professional qualifications. The only way to network with the right people is to indirectly assess them through their online reputation.

\cite{Thung2013} investigates the network structure of social coding in GitHub. In this study, they try to measure the strength of a) relationships among different projects and b) relationship between the developers. For this, they create network graphs of 30,000 projects randomly sampled from a set of 1,00,000 projects returned by the GitHub API and then randomly select 30,000 developers from these projects. This data is used to create two networks - A project-project network and a developer-developer network. In the project-project network each node of the network is a project and two projects are connected if they have at least one common developer. A weight is associated to each edge which signifies the number of developers that work on both the projects. In similar way the developer-developer network is constructed where nodes represent the developers and an edge is drawn between two developers when they work together on at least one project. Weight in this network represents the number of projects the two developers have worked upon together. These graphs are then analysed on three network characteristics - 1) Node degree 2) Network diameter and 3) Average shortest path using methodologies mentioned in \cite{Surian2010}. The authors also use the PageRank algorithm on these networks to find out the most influential projects and developers. The results show that the project networks are more interconnected than human networks and that social coding enables substantially more collaboration among developers.

In \cite{Dabbish2012}, the effect and value of ``transparency'' on people's inferences and decisions is analysed. This study is based on \cite{weiner2013}, in which it is claimed that ``awareness about the activities of user behavior on a social website enables other users on the network to draw inferences''. They performed a series of 24 semi-structured interviews with GitHub users where they were asked about the inferences they make about a user or a project and whether these inferences influence the projects or the people they follow. As expected, a rich set of inferences were made mostly based on 4 visual cues namely volume of activity, sequence of actions over time, attention to artifacts and people and the detail of information about an action. Each of these cues revealed a different type of information to the user making the inferences. For example, one of the users said that commits on a project revealed direction and intent of the contributor. \cite{Shen2011} considers ``homophily'' as another important attachment logic developers use when networking. The term ``homophily'' refers to the tendency for people to be attracted to other people having similar beliefs, attitude and personal characteristics. In contrast to traditional systems where people tend to self categorize with regard to race, gender, socio-economic status etc. these distinctions become much less prevalent and visible on collaborative OSS development platforms. In OSS communities factors such as leadership roles, popularity, seniority and contributions shape a person's online identity which later serve as common grounds to network with other people. For instance, developers are likely to connect with those who have similar level of performance and experience. It is evident from all these studies that online social factors do play a major role in social coding and in depth analysis of these factors can help us to answer several questions which would assist and motivate developers to contribute and learn from these communities.

GitHub 

GitHub \cite{GitHub} is a social coding website that works on Git revision control system. It provides social network functionalities such as feeds, followers and a social network graph to represent collaborations between developers. We will briefly describe typical functionalities GitHub offers and later would use them to answer the questions our study poses. GitHub consists of two types of pages - First, when a user logs into his account, he is directed to his home page. This page shows the user profile which is visible to other developers. Second type of page on GitHub is the project or the repository page which contains all the information regarding a particular project. Both these pages have several functionalities associated with them. The user profile shows number of followers, users being followed, repositories, contributions, contribution calendar and activity log. The ``following'' functionality in Github allows users to follow other developers online i.e. users get notifications about their activity. For example, projects they are working on and people they are connecting with.

The ``contribution'' tab in user profile displays the contribution of the user in projects. A contribution to a project could be a commit, opening an issue or proposing a ``Pull Request''. In sum, the user page creates an identity for the developer, projects he/she is working on and contributions he/she has made to other projects. The project page consists of stars, watchers, issues, forks, branches, programming languages, wikis, pulse, graphs, networks and contributors. Stars and watchers are features that allow users bookmark repositories and receive notifications about any activity reported for it. The issues section allows users to actively report and fix bugs. Issues can be created and viewed by any user on a public repository. Wikis are used to communicate and share notes while pulse, graphs, networks and contributors are used to keep a track of collaborators and activity on the project.
