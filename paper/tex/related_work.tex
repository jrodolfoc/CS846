\section{Related Work}

Pujol et al. introduce SPAR \cite{Pujol12} as a middle-ware solution that
provides partitioning and replication to enforce local semantics. The graph
model that they exploit is one in which a node has many attributes, and those
attributes get updated. This model is in contrast to the work presented in
LinkBench \cite{Armstrong13}, where nodes are simple (i.e. have only one data
field) and updates are represented by creating new nodes with corresponding
edges. As such, SPAR is a poor candidate for storing OSN data that follows the
storage model used by Facebook, since local semantics can not scale to that
volume of neighbours. Local semantics is also not needed, since a big volume of
neighbouring data becomes cold in time.

Mondal et al. \cite{Mondal12} improve on the model introduced by SPAR, by only
requiring approximately $\tau \in [0, 1]$ of the total number of neighbours to
be local. They also exploit the read/write patterns of nodes to decide when to
replicate nodes in a different partition. However, they encounter a similar
problem as the one in SPAR; their model assumes that nodes have a variable write
frequency. They use this property to cluster nodes together in replication
groups. This clustering of nodes removes the underlying graph structure,
especially when all nodes have an equal write frequency (i.e. mostly only the
creation of the node). This results in clusters of nodes that are replicated
together, but which are not necessarily accessed together, thereby still
requiring remote traversals in order to answer queries.
