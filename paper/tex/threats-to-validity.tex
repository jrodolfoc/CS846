\section{Data and Queries}

We make use of LinkBench for generating our graph.  The probability of a node
having some number of edges follows a Zipfian distribution. This means that the
probability of a node having some degree decreases exponentially as the degree
increases.

% read only
% reads of nodes are Zipfian
We re-use their application to generate our workloads. The workloads
that LinkBench see are after cache. As a result, we attempt to approach a
pre-cache workload by adjusting the parameters to LinkBench's distributions.
Second, we assume a read only workload, because TAO and LinkBench observed
primarily reads. This allows us to focus on cache optimizations. The focus of
this paper is not cache invalidation and isolation. The probability of a user
being read some number of times also follows a Zipfian distribution. What this
means is that the probability of a node being read some number of times
decrease exponentially with respect to the number of reads.

% 2-hop query
% random walk queries
% return all nodes
Since photos, status updates, and people are nodes in the graph, queries requesting
all of the photos that a person's friends posted become two hop. One hop to
determine the person's friends, and a second hop to determine their friends' photos.
As a result, we modify the LinkBench workload generator to perform 2-hop queries.
LinkBench only generates 0-hop queries like: getting a node, getting the edges of a
node, and getting an edge. Additionally, we include random walk queries that
represent a user following edges starting from their starting node.

% Result limit
% no history
LinkBench limits the number of nodes returned based on their timestamp. The
reason is that a person would be interested in the most recent events. Since
less than 3\% of queries returned more than 100 vertices, we limit the number of
vertices returned to 100. Secondly, after retrieving the initial vertices, the
probability of reading historically for old vertices was 0.96\%. We removed
these from our workload because a small amount of queries asked for it.
