
\section{Threats to Validity}
\label{sec:threats}

As stated on Section \ref{sec:collection}, we were able to collect only a fraction of the data that GitHub makes publicly available through their API. The limitation of 5,000 requests per hour per user, while generous, limited our progress. We believe that this is alleviated by the fact that our sample is of considerable size (20\% of the users and 20\% of the repositories). We also made sure we included the most popular users and repositories on our sample, by selecting them as a seed for the crawler to recursively traverse the user/repository base of GitHub.

A major threat to our study is the influence of external factors as a driver for collaboration on users. Sites like \cite{SourceGraph} or \cite{OhLoh} act as search engines for open source projects and could potentially influence developers' contributions. These search engines use their own indexes and criteria to display results to users, possibly hiding the information of popularity of users and repositories. We believe that this is partially alleviated by the fact that these sites only work on a subset of the repositories. For example, in the case of \cite{SourceGraph}, they only index Python, Go, Ruby and Javascript repositories.

Finally, we were limited by the information provided by the GitHub API. As an example, we wanted to conduct an historical analysis on popular repositories and determine when did contributors join a repository, i.e. the date of a developer's first contribution. We hope this is solved on future releases of the API, but as of version 3.0, this feature is missing.
