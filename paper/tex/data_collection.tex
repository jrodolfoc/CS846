
\section{Data Collection}
\label{sec:collection}

We used GitHub API v3 \cite{GitHubAPI} to collect the data. The advantages of such API is that it provides a more structured and straightforward access to the data. The disadvantage is that it only allows 5,000 queries per hour per user. An alternative to the API would be to create our own web crawler/scraper, with the risk of our IP getting banned by GitHub. We eventually found a way to paralelize the data collection phase, with the use of several user accounts. We built a tool that, given an initial set of users, would recursively fetch his/her followers and repositories. For the initial set, we chose the one thousand most popular users, plus some chosen randomly. The same could not be done for the repositories. There is no unified criteria for selecting the most popular repositories: number of forks, number of contributors, number of watchers and so on. So, besides the ones that belong to the popular users, we included random sets of repositories, ordered by different criteria.
\begin{table}
\centering
\begin{tabular}{ | l | c | }
	\hline
	Users & 618,353 \\ \hline
	Repositories & 868,177 \\ \hline
\end{tabular}
\caption{Data collected from GitHub API v3}
\label{tbl:datacollected}
\end{table}

The total amount of users and repositories collected from the API is given in Table \ref{tbl:datacollected}. Regarding Q2, Table \ref{tbl:topusers} provides an overview of the typical user in GitHub. For comparison purposes, we decided to group the data by the ranking of the user. This shows, for example, that the average number of followers between the Top 10 users is almost three times that of the Top 100 users. An interesting trend shown here is that the top users are also following a considerable amount of users, with the top 1K users following an average of 121 users, compared to the 2.8 followings that a regular user has.
\begin{table}
\centering
\begin{tabular}{ | c | c | c | c | }
	\hline
	Users & Repositories & Followers & Followings \\ \hline
	Top 10 & 93.4 & 10,231.4 & 81.3 \\ \hline
	Top 100 & 129.1 & 3,203.2 & 70.6 \\ \hline
	Top 1K & 70.8 & 744.5 & 121.0 \\ \hline
	Top 10K & 45.8 & 147.7 & 49.2 \\ \hline
	All & 5.1 & 4.1 & 2.8 \\ \hline
\end{tabular}
\caption{Average repositories, followers and followings of users in GitHub}
\label{tbl:topusers}
\end{table}

Finally, Table \ref{tbl:topcompanies} shows the companies 
\begin{table}
\centering
\begin{tabular}{ | l | c | c | c | c | }
	\hline
	\- & \multicolumn{2}{|c|}{Top 10} & \multicolumn{2}{|c|}{Top 1K} \\ \hline
	\multicolumn{1}{|c|}{Company} & Users & Followers & Users & Followers \\ \hline
	GitHub & 4 & 47,976 & 41 & 81,685 \\ \hline
	Google & 2 & 17,766 & 23 & 30,044 \\ \hline
	Khan Academy & 1 & 6,674 & 1 & 6,674 \\ \hline
	Linux Foundation & 1 & 15,913 & 1 & 15,913 \\ \hline
	Segment.io & 1 & 8,033 & 3 & 8,841 \\ \hline
	Basecamp & - & - & 4 & 7,880 \\ \hline
	Heroku & - & - & 7 & 8,390 \\ \hline
	Twitter & - & - & 11 & 7,701 \\ \hline
	Others & - & - & 556 & 351,946 \\ \hline
	Not Specified & 1 & 5,952 & 353 & 225,466 \\ \hline
\end{tabular}
\caption{Number of users that a company has between the Top 10 and Top 1K. It shows, for example, that 2 of the users in the top 10 are working for Google and they contribute }
\label{tbl:topcompanies}
\end{table}
