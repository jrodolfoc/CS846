
\section{Introduction}

With the advent of Web 2.0 online collaboration has become not only a part of internet experience but rather a necessity. People today are not just passive viewers of the online content but rather active contributors. Social networking sites such as Facebook and Linkedin allow users to create both, their personal and professional identities respectively. Further, online communities such as blogs and Wikis allow people to express opinions, carry out discussions, ask questions and answer queries, among other things. In addition, media sharing and hosting websites such as YouTube ensure fast propagation and easy accessibility of digital content. Clearly, digital collaborations have revolutionized the way information is generated (i.e. through multiple sources), stored (social graphs and complex data types) and accessed (fast and ubiquitous) using just a web browser. In this paper we will particularly focus on collaborations between software developers, “Social Coding” and Open Source Software development. We will discuss how these collaborations function by analysing one of the most popular social coding website -GitHub. We would then conduct a study to assess various factors which govern the dynamics of such kind of collaborations.
Recently, platforms such as SourceForge, GitHub, BitBucket, CodePlex have experienced a rise in popularity. These are also referred to as social coding platforms i.e. development environment that encourages formal and informal collaboration on software projects providing opportunities for discussion and sharing. Open source software in particular has experienced a rise in popularity. These open source projects are licensed to be freely modified and distributed and hence are often developed publically where multiple developers collaborate to add or modify functionality and remove bugs. This new style of development is driven by a rather different set of factors than traditional development. A study conducted by John et al [1] shows that Open source development follows a very modular approach where emphasis is on implementation. Unlike in traditional software development where a requirement specification document specifies design and approach, in Open source, design is a secondary concern which is consistently shaped throughout the development lifecycle via discussions (blogs) and mailing lists and is rather a byproduct of the source code. In addition to this there is also a difference how developers interact on social coding platforms. One of the most important characteristic of OSS development is voluntary contribution [2] without monetary incentives . While proprietary software is a subject to investment and returns, open source communities encourage transparency [3] , collaboration and sharing of ideas. From an economic standpoint, generation of original work takes time and effort in OSS but is compensated in low distribution cost which are almost negligible [4] . Most importantly, open source embraces creative freedom, collaboration, voluntary participation and encourages derivative works at minimal or zero access costs. Moreover, these collaborative platforms do not follow a traditional organizational hierarchy. Rather, the social graph structure of these projects introduces several different factors which guide and shape these collaborations. For example, a studies [4] [5] show that reputation of OSS developers plays an important role in attracting contributors to their projects. This is because, in such a setting where there is absence of personal knowledge about other developers for example their educational and professional qualifications, the only way to network with the right people is to indirectly assess them through their online reputation. This social aspect of OSS development provides insight into dynamics of collaborations and open interesting opportunities for research. Several studies throw light on what drives people towards OSS. Some of these factors include personal need for software [], creative stimulation [], learning [], gift-giving intentions [] and significant returns on contribution []. 
