
\section{Conclusions}

Scaling graph databases can be challenging, especially when dealing with highly
interconnected graphs.  We explored different, well-known caching algorithms
that allowed us to replicate data across partitions based on Zipfian read
workloads.  We also introduced a novel caching algorithm, Local Connected Cache
(LCC), as a way to exploit the underlying nature of graph data. LCC is a
multi-level LRU caching algorithm that assigns scores to vertices based on how
well connected they are to other local vertices on a given partition.

Although this algorithm highly favors immediate (1-hop) neighbours, we
showed that it does not perform well for k-hop neighbours and that is highly
dependent on the query access model that the underlying database implements.
We believe LCC will improve performance significantly if allowed to work with
intermediate caches, a direction we wish to pursue on future versions of the
platform.

We then compared two other caching algorithms, LRU and ARC, and showed
that ARC provides better performance in all three metrics that we collected.
