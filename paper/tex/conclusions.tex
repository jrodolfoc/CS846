
\section{Conclusions}
\label{sec:conclusions}

Understanding what motivates a user into following a project can be helpful for project owners interested in building a community of developers that contribute to its improvement. It is also important for the developer who is looking for a project to contribute. The rise in popularity of public code repositories provide a large amount of opportunities for developers to share ideas, share common interests and to improve their skills. But, this large amount of options can also be overwhelming. Yet, somehow, some of the projects seem to be more attractive to developer. While some projects seem to grow on popularity simply because they are already popular, some counterexamples can be found in which a project quickly gathers a significant following.

Our study focused on studying this phenomenon. In order to explain it, we came up with four hypothesis: (i) the popularity of the project grows because of the popularity of the owner of the project; (ii) the popularity of the project grows because the number of popular contributors working on it; (iii) the popularity of the project is affected by the programming language; (iv) the popularity of the project is determined by the category that it belongs to. Hypothesis (i) and (ii) showed no correlation at all, while hypothesis (iii) showed some correlation between the popularity of the programming language and the number of projects amongst the most popular ones. Finally, hypothesis (iv) showed a high number of popular projects being frameworks or widgets/plugins. But the small sample size (100 projects) makes it difficult to reason about the validity of the hypothesis.

Future work could include the use of machine learning algorithms to automate the categorization of projects, in order to have a much larger sample size and validate/invalidate hypothesis (iv).